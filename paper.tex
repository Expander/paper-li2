\documentclass[10pt,DIV16,twocolumn]{scrartcl}

\usepackage[utf8x]{inputenc}
\usepackage[T1]{fontenc}
\usepackage{lmodern}
\usepackage{amsmath}
\usepackage[german]{authblk}
\usepackage[format=plain,labelfont={bf}]{caption}
\usepackage{listings}
\usepackage{hyperref}

\newcommand{\mytitle}{Notes on the high-performance implementation of special functions}
\newcommand{\myauthor}{Alexander Voigt}

% source code highlighting
\lstset{breaklines=true,
  breakatwhitespace=true,
  % numbers=left,
  % stepnumber=1,
  basicstyle=\ttfamily\lst@ifdisplaystyle\footnotesize\fi, % code block in smaller font
  commentstyle=\ttfamily,
  prebreak={\textbackslash},
  breakindent=10pt,
  breakautoindent=false,
  showspaces=false,
  showstringspaces=false,
  frame=shadowbox,
  rulesep=0.1em,
  abovecaptionskip=0em,
  aboveskip=1.5em,
  belowcaptionskip=0.5em,
  belowskip=1em,
  keywordstyle=\color{blue},
  stringstyle=\color{red},
  commentstyle=\color{green},
}

\hypersetup{
  pdftitle   = {\mytitle},
  pdfauthor  = {\myauthor},
  colorlinks = true,
  linkcolor  = blue,
  citecolor  = blue,
  urlcolor   = blue,
  linkcolor  = blue
}

\title{\mytitle}
\author{\myauthor}
\affil{Fachbereich Energie und Biotechnologie, Hochschule Flensburg, Kanzleistraße 91--93, 24943 Flensburg, Germany}
\date{\today}

\begin{document}
\maketitle

\section*{Abstract}

\section{Introduction}

\cite{polylogarithm}

\section{Comparison of algorithms}

As an example we consider the function $\log(1+x)$, which has the
following Taylor expansion around $x=0$ for $-1 < x \leq 1$:
%
\begin{align}
  \log(1+x) \approx x - \frac{x^2}{2} + \frac{x^3}{3} - \frac{x^4}{4} + \frac{x^5}{5} - \frac{x^6}{6} + \frac{x^7}{7},
\end{align}
%
where terms of $O(x^8)$ have been omitted.
In C this function could be implemented as follows:
%
\begin{lstlisting}[language=C]
double log1p_naive(double x) {
   return x - x*x/2 + x*x*x/3 - x*x*x*x/4
      + x*x*x*x*x/5 - x*x*x*x*x*x/6
      + x*x*x*x*x*x*x/7;
}
\end{lstlisting}
%
This naive implementation requires 26 floating-point multiplications
and 6 floating-point additions.\footnote{Modern compilers usually
  rewrite expensive floating-point divisions by a constant, such as
  $x/2$, as less expensive floating-point multiplication by a
  constant, for example as $0.5x$.}

As is well known, the performante of this implementation can be
improved by applying Horner's scheme, which reduces the number of
expensive floating-point multiplications to 7 and requires 6
additions:
%
\begin{align}
  \log(1+x) \approx{}& x \Bigg\{1 + x \Bigg[-\frac{1}{2} + x \Bigg(\frac{1}{3} + x \Bigg\{-\frac{1}{4}  \notag \\
  &+ x \Bigg[\frac{1}{5} + x\Bigg(-\frac{1}{6} + \frac{x}{7}\Bigg)\Bigg]\Bigg\}\Bigg)\Bigg]\Bigg\}.
\end{align}
%
A C implementation using Horner's scheme could read
%
\begin{lstlisting}[language=C]
double log1p_horner(double x) {
   return x*(1 + x*(-0.5 + x*(1./3
     + x*(-0.25 + x*(0.2
     + x*(-1./6 + x/7))))));
}
\end{lstlisting}

Several modern CPU architectures support instruction-level parallelism
(ILP), which allows for the execution of multiple instructions at the
same time.  To make use of ILP to reduce the execution time of the
function, one could use Estrin's scheme \cite{estrin} to rewrite the
function as
%
\begin{align}
  \log(1+x) &\approx x + \left(\frac{x}{3}-\frac{1}{2}\right)
    y+\left(\left(\frac{x}{7}-\frac{1}{6}\right)
    y+\frac{x}{5}-\frac{1}{4}\right) z,
\end{align}
%
where $y=x^2$ and $z=y^2$.  This form requires 8 floating-point
multiplications and 6 additions, i.e.\ more multiplications than
Horner's scheme.  However, in Estrin's form several terms can be
executed in parallel, leading to a potential speed-up.  A C
implementation using Estrin's scheme could read:
%
\begin{lstlisting}[language=C]
double log1p_estrin(double x) {
   const double y = x*x;
   const double z = y*y;
   return x + (x/3 - 0.5)*y
      + (y*(-1./6 + x/7) + x/5 - 0.25)*z;
}
\end{lstlisting}

\section{Conclusions}

\bibliographystyle{ieeetr}
\bibliography{paper}

\end{document}
